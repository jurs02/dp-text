\chapter{Sentimentr study guide}
\sloppy
\section*{Klasifikace sentimentu nástrojem \emph{Sentimentr}}

Z hlediska úlohy jde o standardní klasifikaci sentimentu jako pozitivního nebo negativního.
Vhodné zejména pro dokumenty typu (produktových apod.) recenzí, kde lze za určitých okolností chápat původce a čas informací i hodnocený aspekt jako nezajímavé nebo implicitní hodnoty.

Přístup je založen na slovníkovém přístupu, přičemž je ale původní polarita pozitivních a negativních slov modifikována čtyřmi specifickými typy slov vyskytujícími se v jejich kontextu -- jde o:
\begin{itemize}
\item negátory
\item zesilovače (``amplifiers'')
\item zeslabovače (``de-amplifiers'')
\item odporovací spojky (``adversative conjuctions'').
\end{itemize}

Zdrojové kódy i manuál jsou na \url{https://github.com/trinker/sentimentr}.

\subsection*{Sentiment věty a celého textu}

Metoda se aplikuje na text složený z vět $s_i$, které jsou posloupnostmi slov $w_{ij}$ (resp. výskytů slov, protože některá slova se mohou ve větě opakovat):
\[
s_i = (w_{i1}, w_{i2}, ..., w_{in})
\]
Dále se budeme zabývat jen aplikací metody na jednotlivou větu $s = (w_1, ..., w_n)$; index věty již neuvažujeme.

Metoda nalezne ve větě $s$ všechny výskyty slov ze základního slovníku $Pol$ obsahujícího polarizovaná slova.
% $w_i \in Pol$ (zde i dále pro zjednodušení chápejme výrazy typu $w \exists Dict$ na výskyt slova $w_i$ jako ``slovo vyskytující se na pozici $w_i$ je obsaženo ve slovníku  $Dict$'').
Jejich polarita $pol(w_i)$ může být nastavena buď na $+1$ vs. $-1$ (pokud slovník obsahuje jen prosté seznamy pozitivních a negativních slov), nebo pomocí specifických vah jednotlivých slov.

Sentiment věty $s = (w_1, ..., w_n)$ se pak počítá jako součet sentimentů všech výskytů slov ze slovníku $Pol$ relativně k délce věty (resp. dle návrhu autora metody, k její odmocnině):
\[
sentiment(s)=\frac{\sum_{w_{i}\in Pol} sentiment(w_i)}{\sqrt{|w_i|}}
\]

Sentiment se počítá pro každou větu samostatně; agregaci na úroveň celého textu lze provést prostým průměrováním, případně jako vážený průměr, který může např. s větší vahou započítávat záporné hodnoty (toto v systému zajišťuje volba \texttt{average\_weighted\_mixed\_sentiment}).

Nyní tedy potřebujeme jen vědět, jak metoda vypočítá sentiment jednotlivých polarizovaných slov z věty, $sentiment(w_i)$.

\subsection*{Polarizované kontextové shluky}

Původní polaritu každého polarizovaného slova $w_i$ je nutno upravit na základě analýzy tzv. \emph{polarizovaného kontextového shluku} $pcc(w_i)$.
Výchozí podobou shluku pro dané polarizované slovo $w_i$ (vzhledem ke kterému je kontext vytvářen) je posloupnost výskytů slov
\[
pcc^{ini}(w_i)=(w_{i-b}, ..., w_{i-1}, w_i, w_{i+1}, ..., w_{i+a})
\]
kde $a$ a $b$ jsou parametry označované jako \texttt{n.before} a \texttt{n.after}; jako jejich hodnoty jsou používány $a=4$ a $b=2$.
Dále je aplikován slovník $Pau$ obsahující ``slova'' vyjadřující ``pauzu'' (ve skutečnosti nemusí jít o slova, ale např. o znaky jako je středník): pokud je polarizované slovo od modifikátoru odděleno takovým slovem, modifikátor na něj nemá vliv. 
Formálně, každý výchozí shluk $pcc^{ini}(w_i)$ je upraven na redukovaný shluk takto:
\[
pcc(w_i) = pcc^{ini}(w_i) \setminus \{w_j \; | \; \exists w_p \in Pau \; \textit{takové že} \; (j \leq p < i \; \vee j \geq p > i) \; \}
\]

\subsection*{Aplikace modifikátorů}

Jádrem metody je výpočet sentimentu jednotlivého slova, který vychází z jeho původní slovníkové polarity, ale upravuje ji na základě modifikátorů vyskytujících se v kontextovém shluku: negátory (resp. lichý počet negátorů) otáčejí polaritu, zatímco amplifikátory ($amp$), deamplifikátory ($deamp$) a odporovací spojky ($advcon$) modifikují intenzitu sentimentu:
%; souhrnný vliv každého typu modifikátoru je ve výpočtu zahrnut jako jeden člen součinu:

\[
sentiment(w_i)=pol(w_i)\cdot (-1)^{neg(w_i)}\cdot (1+amp(w_i)+deamp(w_i)) \cdot advcon(w_i)
\]
\emph{Pozn.: Některé části výpočtu nejsou přepsány zcela podle popisu na} \url{https://github.com/trinker/sentimentr}\emph{, ale intuitivním odhadem, protože v původním popisu buď nedávaly smysl nebo některá souvislost chyběla.}

\begin{enumerate}

\item 
U negátorů, ze slovníku $Neg$, se jednoduše zjistí jejich lichý nebo sudý počet. \\
\[
neg(w_i)=|w_j \in pcc(w_i)\cap Neg| \; mod \; 2
\]
\emph{Pozn.: 
Zápis množinové operace je zde, pro zjednodušení, vyjádřen matematicky nepřesně. 
$pcc(w_i)$ je sekvence s možností opakování (tj. multimnožina), zatímco $Neg$ je neuspořádaná množina (bez opakování).
Jako jejich průnik zde chápeme podsekvenci obsahující ty prvky z $pcc(w_i)$, které jsou i v množině $Neg$.
Stejná konvence je i u ostatních slovníků/modifikátorů níže. \\ 
Dále, binarizace negátorového faktoru pomocí funkce zbytku ($mod$) není potřebná kvůli celkovému výpočtu z předchozího vzorce, ale kvůli použití tohoto faktoru v rámci de/amplifikace, viz níže.}

\item 
Amplifikace se vyjádří jako:
\[
amp(w_i)=(1-neg(w_i)) \cdot ad \cdot |w_j \in pcc(w_i)\cap Amp|
\]
kde $Amp$ je slovník amplifikátorů a $ad$ společná konstanta vyjadřující váhu amplifikátorů a deamplifikátorů, empiricky nastavená na 0,85.
Pokud je ve shluku lichý počet negací, vliv amplifikátorů se v tomto vzorci ``vypne'' (resp. naopak se započítává níže jako deamplifikátor!), v opačném případě se každý amplifikátor započítává úměrně konstantě $ad$.

\item 
Deamplifikace se vyjádří jako:
\[
deamp(w_i)= ad \cdot (|w_j \in pcc(w_i)\cap Deamp| + neg(w_i) \cdot |w_j \in pcc(w_i)\cap Amp|)
\]
kde $Deamp$ je slovník deamplifikátorů.
Pokud je tedy ve shluku lichý počet negací, amplifikátory se započívávají tak, jako by byly deamplifikátory.
Deamplifikace je opět úměrná konstantě $ad$.
\item
Vliv odporovacích spojek je vyjádřen jako
\begin{multline*}
advcon(w_i)= \\
1\;+\;ac\; \cdot \; (|w_j \in pcc(w_i) \cap AdvCon, j<i| \; - \; |w_j \in pcc(w_i)\cap AdvCon, j>i|)
\end{multline*}
kde $AdvCon$ je slovník odporovacích spojek a $ac$ je konstanta empiricky nastavená na 0,85.
Spojka za polarizovaným slovem je tedy zeslabuje, spojka před polarizovaným slovem je naopak posiluje (intuice je, že ``odporující'' vyjádření chce pisatel oproti původnímu tvrzení zdůraznit). 

\end{enumerate}

\flushright{ Zpracoval V. Svátek, 2. 5. 2019}