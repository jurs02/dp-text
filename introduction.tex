\chapter*{Introduction}
\addcontentsline{toc}{chapter}{Introduction}

The aim of this thesis is to create a system for extracting opinions and sentiment from conference paper reviews. In other words the system will allow to automatically determine the opinion of the author of the review on different aspects of the reviewed paper. 

Because each submitted paper is going to be reviewed by many people, and because each conference has its own structure of the review form as well as different set of criteria it’s very difficult to quickly get an idea of the quality of the submitted work. Being able to get a general idea of how good a paper is as well as quickly assessing what are the strong and weak parts of a submission is especially important for meta-reviewers during the discussion periods.

Extracting the opinions of the reviewer on a paper, mapping it onto a unified set of criteria and transforming it into a numerical value could significantly simplify the process of submission acceptance as well as provide a way to compare reviews of the same paper across different conferences and reviewers.

The system created here could also serve as a base for a larger review management system that is able to generate a visual metaphor of each  review that reflects different review metrics. Visual images are faster and easier to understand than written text, therefore it should make the meta-analysis more comfortable and effective.

In order to implement such a system, it’s necessary to create a set of review metrics to which the fields of different conference review forms will be mapped. Then apply a technique of information extraction to get the opinion of the reviewer on each of these metrics. In order to then get a numeric  evaluation of the sentiment of the specific opinion a sentiment analysis technique has to be used. 
